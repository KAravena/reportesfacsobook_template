% --- title-pdf.tex robusto (sin quiebres peligrosos) ---
\makeatletter
\providecommand{\subtitle}[1]{\gdef\@subtitle{#1}}
\providecommand{\@subtitle}{}

% Impresores seguros (evitan espacios/saltos indebidos)
\newcommand{\PrintTitle}{%
  {\sffamily\bfseries\fontsize{22pt}{26pt}\selectfont \@title\par}%
}
\newcommand{\PrintSubtitle}{%
  \begingroup
  \edef\temp{\detokenize{\@subtitle}}%
  \ifx\temp\empty\relax
    % sin subtítulo
  \else
    {\sffamily\large\itshape \@subtitle\par}%
  \fi
  \endgroup
}
\newcommand{\PrintAuthor}{%
  {\normalsize \@author\par}%
}
\newcommand{\PrintDate}{%
  {\small \@date\par}%
}
\makeatother

\begin{titlepage}
\thispagestyle{empty}
\begin{center}
\vspace*{10mm}

% Logo institucional
\includegraphics[width=0.30\textwidth]{assets/cover.png}\par
\vspace{12mm}

% Título y subtítulo
\PrintTitle
\vspace{4mm}
\PrintSubtitle

\vspace{16mm}
% Autor y fecha
\PrintAuthor
\vspace{2mm}
\PrintDate

\vspace{8mm}
{\small Facultad de Ciencias Sociales — Universidad de Chile \par}

\vfill
\rule{.6\textwidth}{0.6pt}\par
{\footnotesize Plantilla Quarto — Reportes FACSO\par}

\end{center}
\end{titlepage}

% Preliminares (números romanos) y estilo simple
\frontmatter
\pagestyle{plain}

\setcounter{tocdepth}{2} % (o 1/3 según prefieras)