% ---------- Paquetes tipograficos y microajustes ----------
\usepackage{microtype}        % mejor interletrado y justificado
\usepackage{csquotes}         % comillas tipograficas
\usepackage{enumitem}         % listas compactas
\setlist{itemsep=.2em, topsep=.2em}

% ---------- KOMA-Script: estilo de titulos y espaciado ----------
\KOMAoptions{
  headings=big,
  parskip=half,
  fontsize=11pt,
  appendixprefix=true
}
\setlength{\parindent}{1.5em}
\setlength{\parskip}{0.6em}
\usepackage{indentfirst}

% ---------- Encabezados y pies (scrlayer-scrpage) ----------
\usepackage[automark,headsepline]{scrlayer-scrpage}
\clearpairofpagestyles
\automark[chapter]{chapter}
\ihead{\pagemark}
\ohead{\itshape\headmark}
\setheadsepline{0.4pt}
\renewcommand*{\chaptermarkformat}{}
\renewcommand*{\chapterpagestyle}{scrheadings}
\pagestyle{scrheadings}
\makeatletter
\let\ps@plain\ps@scrheadings
\makeatother
\cfoot{}

% ---------- Hipervinculos mas sobrios ----------
\usepackage{hyperref}
\hypersetup{
  colorlinks=true,
  linkcolor=[rgb]{0.05,0.2,0.5},
  citecolor=[rgb]{0.05,0.2,0.5},
  urlcolor=[rgb]{0.05,0.2,0.5},
  pdfauthor={\@author},
  pdftitle={\@title}
}

% ---------- Leyendas de figuras/tablas ----------
\usepackage[labelfont=bf,textfont=it]{caption}
\captionsetup{
  skip=8pt
}

% ---------- Tabla de contenidos ----------
\KOMAoptions{toc=graduated}
\RedeclareSectionCommand[tocnumwidth=3em]{chapter}
\RedeclareSectionCommand[tocindent=3.25em,tocnumwidth=2.8em]{section}
\RedeclareSectionCommand[tocindent=6.5em,tocnumwidth=2.5em]{subsection}
\makeatletter
\newcommand*{\tocdotfill}{\leavevmode\leaders\hbox to .6em{\hss.\hss}\hfill}
\makeatother
\RedeclareSectionCommand[toclinefill=\tocdotfill]{chapter}
\RedeclareSectionCommand[toclinefill=\tocdotfill]{section}
\RedeclareSectionCommand[toclinefill=\tocdotfill]{subsection}
\renewcommand*{\contentsname}{Tabla de contenido}
\setkomafont{chapterentry}{\normalfont}
\setkomafont{chapterentrypagenumber}{\normalfont}

% ---------- Viudas/Huerfanas y cortes de pagina ----------
\clubpenalty=10000
\widowpenalty=10000
\displaywidowpenalty=10000

% ---------- Entorno abstract para scrbook ----------
\providecommand{\abstractname}{Resumen}
\makeatletter
\@ifundefined{abstract}{
  \newenvironment{abstract}{
    \cleardoublepage
    \thispagestyle{plain}
    \null\vfill
    \begin{center}
      {\bfseries\Large \abstractname\par}
    \end{center}\vspace{1em}
    \begingroup
  }{
    \par\endgroup
    \vfill\null
    \cleardoublepage
  }
}{}
\makeatother

% ---------- Soporte de subtitulo desde YAML ----------
\makeatletter
\providecommand{\subtitle}[1]{\gdef\@subtitle{#1}}
\providecommand{\@subtitle}{}
\makeatother

% --- Desactivar portada y abstract automaticos de Pandoc (PDF) ---
\AtBeginDocument{\let\maketitle\relax}
\renewenvironment{abstract}{}{}
\providecommand{\appendixname}{}
\providecommand{\appendixtocname}{}
\providecommand{\appendixpagename}{}
\renewcommand*{\appendixname}{Anexo}
\renewcommand*{\appendixtocname}{Anexos}
\renewcommand*{\appendixpagename}{Anexos}
